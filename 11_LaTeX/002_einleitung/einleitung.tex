\chapter{Einleitung} \label{cha:einleitung}
Zu Beginn der Arbeit sollen zunächst die Hintergründe und damit die Motivation für die Auswahl und Bearbeitung des Themas der Masterarbeit aufgezeigt werden. Darauf folgt eine kurze Beschreibung der Aufgabenstellung in Form einer Zielsetzung. 


\section{Motivation} \label{sec:motivation}
Bereits seit einigen Jahren zeichnet sich durch den Einzug moderner Fahrerassistenzsysteme ein Umbruch im individuellen und gesellschaftlichen Umgang mit dem Automobil ab. Als treibende Aspekte für den technologischen Umbruch in der Automobilentwicklung sind die steigende Verkehrssicherheit, der erhöhte Fahrkomfort und eine energieoptimale Fahrzeugführung zu nennen. Diverse Unfallstatistiken zeigen, dass die Anzahl Getöteter und Verletzter im Straßenverkehr zwar stetig sinkt, die Zahlen jedoch seit einigen Jahren stagnieren \cite{DeutscherVerkehrssicherheitsrat.2018}. Die Weiterentwicklung diverser Fahrerassistenzsysteme könnte die Verkehrssicherheit weiterhin erhöhen und damit dazu beitragen, die Zahl der Unfallopfer im Straßenverkehr weiter zu reduzieren. Maßgebend für den Einzug innovativer Fahrerassistenzsysteme bis hin zur Realisierung automatisierter oder autonomer Fahrfunktionen sind die Entwicklungsarbeit und der Innovationstrieb von Hersteller und Zulieferer der Automobilindustrie zu nennen. Aktuell arbeiten alle namhaften Automobilhersteller daran, die Vision vom autonomen Fahren in naher Zukunft Wirklichkeit werden zu lassen.

Die Forschung und Entwicklung im Bereich künftiger Fahrfunktionen und effizienter Mobilität stellt die gesamte Automobilindustrie jedoch vor besondere Herausforderungen. So gilt es nicht nur die Wünsche und Ansprüche der Kunden zu erfassen um Anforderungen technisch zu realisieren, sondern diese vor Markteinführung auch umfangreich zu testen und gegen potentielle Ausfälle oder unerwünschtes Fehlverhalten abzusichern. Darüber hinaus muss der Hersteller über den gesamten Produktlebenszyklus hinweg den vollen Funktionsumfang ohne technische Mängel der sicherheitskritischen Systeme gewährleisten können. Diese Aspekte und der stetig steigende Konkurrenzkampf um Marktanteile unter den Mitbewerbern setzt forschende Unternehmen unter einen enormen Innovationsdruck. Entwicklungsdienstleistungsunternehmen und Ingenieursgesellschaften wie die \emph{Evomotiv GmbH} können mit ihrer Arbeit hier einen entscheidenden Beitrag leisten. Das Unternehmen mit Wurzeln in der Automobilindustrie unterstützt daher eine Vielzahl an Kunden über den gesamten Produktentstehungsprozess hinweg und übernimmt komplexe Technologieprojekte rund um die Fahrzeugentwicklung.\\
Da die Evomotiv GmbH selbst ein hohes Potential in der künftigen Entwicklung innovativer Fahrfunktionen sieht, hat sich das Unternehmen bereits ein breites Knowhow im Bereich der Fahrerassistenzsysteme aufgebaut. Dieses Firmenwissen soll künftig um automatisierte Fahrfunktionen erweitert werden. Durch eigenständige Systemlösungen können Alleinstellungsmerkmale und gegebenenfalls eigene Produkte entstehen. Aufgrund dessen wurde das Projekt \emph{EVObot} ins Leben gerufen. Das erklärte Projektziel ist es, ein Fahrzeug aufzubauen, das als Demonstrator für automatisierte Fahrfunktionen dienen soll. Dazu wurde ein Modellfahrzeug im Maßstab 1:10 aufgebaut, das ausgewählte autonome Fahrfunktionen nach dem Vorbild des studentischen Hochschulwettbewerbs \emph{Carolo-Cup} \cite{TechnischeUniversitatBraunschweig.} bewältigen soll. Das Entwicklungsprojekt soll für das Unternehmen sowohl als Experimentierplattform für Systeme und Funktionen dienen, als auch einen Wissenstransfer und die Ausbildung von Studierenden und Absolventen in Form von Projekt- oder Abschlussarbeiten ermöglichen.

\section{Zielsetzung der Arbeit} \label{sec:aufgabenstellung}
In einer vorangegangenen Abschlussarbeit \cite{Fitzer.31.03.2018} wurde bereits ein Modellfahrzeug im Maßstab 1:10 als Demonstrator aufgebaut und einige automatisierte Fahrfunktionen teilweise umgesetzt. Dazu wurde geeignete Hardware und Sensorik ausgewählt und hierauf vernetzte Softwarefunktionen umgesetzt. Dabei bestand die Anforderung jedoch nicht darin, ein Modellfahrzeug aufzubauen, welches vollständig autonom in seiner realen Umgebung agiert, sondern lediglich einzelne Funktionen in einem gewählten Szenario zu Demonstrationszwecken darzustellen.

Das Ziel der vorliegenden Arbeit ist es nun, dieses Projekt weiterzuführen und den Funktionsumfang neuer Systeme so umzusetzen, dass sich diese vollständig in das bestehende Projekt integrieren. Der Verlauf der vorangegangenen Abschlussarbeit hat gezeigt, dass sich bei der Entwicklung und Umsetzung eines solch komplexen Gesamtsystems durchaus eine Reihe von Herausforderungen ergeben können. So war es während dem Entwicklungsprozess nicht möglich, eine angemessene Visualisierung und Aufzeichnung der Daten zu realisieren. Dies ist jedoch besonders für die Analyse der implementierten Algorithmen und zu Applikation der Systemparameter essenziell. Ebenso bestand keine Möglichkeit, potentielle Fehler aufzudecken und das Systemverhalten während der Laufzeit angemessen auswerten zu können. Daher soll ein Konzept eines Diagnosesystems für eine mögliche Fehlerdiagnose am Demonstratorfahrzeug entworfen werden, die dem Vorbild einer realen Diagnosekommunikation in Fahrzeugsystemen entspricht. Eine strikte Einhaltung der Entwicklungsstandards rund um Diagnosesysteme im automotiven Umfeld steht dabei jedoch nicht im Fokus der Arbeit. Das Diagnosesystem soll durch den Einsatz einer geeigneten Schnittstelle aufgebaut und eine Funktion zur Fehler- und Zustandsüberwachung implementiert werden. Im Anschluss daran sollen die implementierten Teilfunktionen und Algorithmen zur Bildverarbeitung und Fahrzeugregelung hinsichtlich ihrer Tauglichkeit in diesem Projekt analysiert werden. Abschließend sollen weitere mögliche Fahrfunktionen und potenzielle Verbesserungsansätze des Projektes aufgezeigt werden.
Den genannten Arbeiten wird eine umfassende Einarbeitung in den aktuellen Projektstand und eine Grundlagenrecherche hinsichtlich autonomer Fahrfunktionen, Bussystemen der Fahrzeugtechnik und Bildverarbeitungsprozesse vorangestellt.