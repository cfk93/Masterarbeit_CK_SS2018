\chapter{Abkürzungsverzeichnis} \label{vor:abkuerzungsverzeichnis}

% \manualmark
% \markright{Abkürzungsverzeichnis}
% \markleft{Abkürzungsverzeichnis}

% ACHTUNG: Muss von Hand alphateisch sortiert werden!

\renewcommand{\arraystretch}{1.25} % Zeilenhöhe auf Faktor 1.25 erhöhen

\textbf{\textsf{Kurzform}} \hspace{2.5cm} \textbf{\textsf{Bedeutung}}
\vspace{1ex}

\setlength{\itemsep}{-\parsep}	% Ausrichtung am längsten Akronym
\begin{acronym}[MMMMMMMMMM]		% länsgtes Akronym
	\acro{muc}[$\mu$C]{Mikrocontroller}
	\acro{ACK}{Acknowledgement}
	\acro{API}{Application Programming Interface}
	\acro{Bus}{Binary Utility System}
	\acro{CAN}{Controller Area Network}
	\acro{CAPL}{Communication Access Programming Language}
	\acro{CRC}{Cyclic Redundancy Check}
	\acro{CSMA/CA}{Carrier Sense Multiple Access with Collision Avoidance}
	\acro{ECU}{Electronic Control Unit}
	\acro{FPGA}{Field Programmable Gate Array}
	\acro{HSKA}{Hochschule Karlsruhe – Technik und Wirtschaft}
	\acro{ID}{Identifier}
	\acro{IEEE}{Institute of Electrical and Electronics Engineers}
	\acro{ISO}{International Organization for Standardization}
	\acro{KWP}{Keyword Protokoll}
	\acro{LIN}{Local Interconnect Network}
	\acro{MOST}{Media Oriented Systems Transport}
	\acro{OBD}{On-Board-Diagnose}
	\acro{ODX}{Open Diagnostic Data Exchange}
	\acro{OSI}{Open System Interconnection}
	\acro{SG}{Steuergerät}
	\acro{UDS}{Unified Diagnostic Services}
	\acro{UML}{Unified Modeling Language}
	
\end{acronym}

\renewcommand{\arraystretch}{1} % Zeilenhöhe auf Standard zurücksetzen
