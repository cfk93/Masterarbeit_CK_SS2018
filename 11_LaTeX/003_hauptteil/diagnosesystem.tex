\chapter{Diagnosesystem} \label{cha:Diagnosesystem}
\lipsum[1-1]
\section{Konzept zur Fehlerdiagnose} \label{sec.KonzeptDiagnose} % Allgemeines zur Diagnose im Fzg usw., Methodenauswahl, ...

\section{Aufbau einer Diagnosekommunikation} \label{sec:AufbauDiagnosekommunikation} % HW-Anbindung an CAN-Bus, Arduino/Jetson, ...

\section{Implementierung der Diagnosefunktion} \label{sec:ImplementierungDiagnose} % SW-seitige Umsetzung inkl. kompletter Programmablaufplan, Funktionen erklären, Gesamtfunktion in ROS, ... 

\section{Ergebnisbetrachtung} \label{sec:ErgebnisDiagnose} % Zeitkritisch? Echtzeit? Dauer der Verarbeitung, Wiederholungsrate der Sensorsignale, Rechenleistung, Synchronisation, ...
\subsection{Test und Validierung} \label{subsec:TestValidierungDiagnose}
\subsection{Mehrwert der Diagnosefunktion} \label{subsec:MehrwertDiagnose}



