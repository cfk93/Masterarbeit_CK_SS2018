\chapter{Einleitung} \label{cha:einleitung}

\section{Motivation} \label{sec:motivation}
Bereits seit einigen Jahren zeichnet sich durch den Einzug moderner Fahrerassistenzsysteme ein Umbruch im individuellen und gesellschaftlichen Umgang mit dem Automobil ab. Als treibende Aspekte für den technologischen Umbruch in der Automobilentwicklung sind die steigende Verkehrssicherheit, der erhöhte Fahrkomfort und eine energieoptimale Fahrzeugführung zu nennen. Alleine im Jahr 2017 kamen 3.177 Menschen bei Verkehrsunfällen auf deutschen Straßen ums Leben [Quelle].
FAS bieten Potential um Unfälle zu senken. 

Autonomes Fahren könnte einen Umbruch im individuellen und gesellschaftlichen Umgang
mit dem Automobil nach sich ziehen und damit auch Einfluss nehmen auf Verkehr,
Mobilität oder Raumstrukturen.

\section{Zielsetzung der Arbeit} \label{sec:aufgabenstellung}
\lipsum[1-3]