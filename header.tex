%-------------------------------------------------------------------
%  Laden sämtlicher Packages	
%-------------------------------------------------------------------
%

\usepackage[utf8]{inputenc} % Eingabecoidierung der Umlaute nach UTF-8

\usepackage[T1]{fontenc}	% Schriftcodierung der Umlaute in westeuropäischer Form

\usepackage[ngerman]{babel}	% Neue deutsche Rechtschreibung

\usepackage{lmodern}		% Standard-Schriftfamilie Latin Modern

\usepackage{parskip}		% Absatztrennung nach europäischer Norm

\usepackage{makeidx}		% Indexerstellung

\usepackage{lipsum}			% Lorem Ipsum

\usepackage{graphicx}		% Einbetten von Grafiken
\usepackage{epstopdf} 

%\usepackage[hidelinks]{hyperref}		% Hyperlinks innerhalb des Dokumentes

\usepackage{longtable}		% Für Tabellen, die sich über mehrere Seiten erstrecken

%\usepackage{natbib}
%\bibliographystyle{unsrt}

\usepackage{url}
\usepackage[isbn=false, doi=false, backend=bibtex, sorting=none]{biblatex}
\setlength\bibitemsep{0.5\baselineskip}
\DefineBibliographyStrings{german}{% 
	urlseen = {geprüft am}, 
}
%\nocite{*}
\addbibresource{005_bib/literatur2.bib}


\usepackage{amsmath}
\usepackage{amssymb}

\usepackage[printonlyused]{acronym}

\usepackage{geometry}

\usepackage{booktabs}

\usepackage{tabulary}		% Für schönere Tabellen mit bestimmter Breite
\usepackage{colortbl}		% Für farbig hilterlegte Zellen in einer Tabelle

\usepackage{textcomp}

\usepackage{pdfpages}		%zum Einbinden von pdf-Dokumenten

\usepackage[colorlinks=true, linkcolor=black, pdfdisplaydoctitle=true,citecolor=black]{hyperref} %für pdf-Metadaten
\hypersetup{
	pdftitle    = {Weiterentwicklung eines autonom fahrenden Demonstrators für Fahrerassistenzsysteme},
	pdfsubject  = {Masterthesis},
	pdfauthor   = {Christof Kary},
	pdfpagemode  = UseOutlines, % Anzeige Bookmarks
	bookmarksopen = true,     % Anzeige Ebenen
	bookmarksnumbered = false, % Anzeige Abschnittsnummern  
	pdfstartpage = {1},        % Startseite, hilfreich mit pdf
	urlcolor=black
}

\usepackage{tikz}
\usepackage{tikz-uml} % Für UML-Diagramme

%-------------------------------------------------------------------
%  Pfade der Dateien setzen
%-------------------------------------------------------------------
%
\graphicspath{{006_img/}}

%-------------------------------------------------------------------
%  Sonstige Befehle	
%-------------------------------------------------------------------
%
\makeindex

\linespread{1.1} % Zeilenabstand für Dokument